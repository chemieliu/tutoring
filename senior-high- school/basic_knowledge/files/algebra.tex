\documentclass[../Main.tex]{subfiles}
\usepackage{pgfplots}
\pgfplotsset{compat=1.18}
% \setcounter{secnumdepth}{3} % 让三级标题有编号
% \setcounter{tocdepth}{3}    % 目录显示到三级标题

\begin{document}
\chapter{Large Numbers}

\intro{
    主要介绍大数及其读写。
}
\section{知识}

个(一)、十、百、千、万……亿,十亿,百亿,千亿都是\textbf{计数单位},每相邻两个计数单位之间的进率都是十。把这些计数单位按照一定的顺序排列起来,它们所占的位置可作\textbf{数位}。(如下表)

{
\zihao{5}
\begin{center}
    \begin{tabular}{|c|*{12}{c|}}
        \hline
        数级   & \multicolumn{4}{c|}{亿级} & \multicolumn{4}{c|}{万级} & \multicolumn{4}{c|}{个级}                                                 \\
        \hline
        数位   & 千亿位                     & 百亿位                     & 十亿位                     & 亿位 & 千万位 & 百万位 & 十万位 & 万位 & 千位 & 百位 & 十位 & 个位 \\
        \hline
        计数数位 & 千亿                      & 百亿                      & 十亿                      & 亿  & 千万  & 百万  & 十万  & 万  & 千  & 百  & 十  & 个  \\
        \hline
    \end{tabular}
\end{center}
}


\vspace{1em}
\textbf{知识回眸}

\vspace{1em}
\textbf{经典重现}

\textbf{例1} \quad 550808080 这个数中,两个 8 分别表示 8 个(\underline{\hspace{3em}})和 8 个(\underline{\hspace{3em}}),两个 5 分别表示 5 个(\underline{\hspace{3em}})和 5 个(\underline{\hspace{3em}})。

解:整数数位从右向左,每四位为一级,依次为个级,万级,亿级等。\\
550808080 分级可划分为 55:0808:0800。\\
从右起第一个 8 在百位,表示 8 个百;第二个 8 在十万位,表示 8 个十万;\\
第一个 5 在千万位上,表示 5 个千万;第二个 5 在十亿位上,表示 5 个十亿。

\vspace{1em}
\textbf{例2} \quad 先分级,再读出下面各数。\\
35000000 \quad 20220000 \quad 7807800 \quad 6000066 \quad 7800780 \quad 8080800











\let\cleardoublepage\clearpage





\end{document}
