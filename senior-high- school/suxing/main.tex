\documentclass[A4,12pt,twoside]{book}
\usepackage[UTF8]{ctex}  % 中文支持
\usepackage{amd}
\usepackage{amsmath}
\usepackage{pgfplots}
\usepackage{exsheets}
\usetikzlibrary{calc}
% ÷\usepackage{tkz-euclide} % 几何绘图宏包
% \usetkzobj{all}
\usetikzlibrary{angles,quotes} % 引入角度库
\usepackage{tasks} % 横向排列选项
\usepackage{tikz-3dplot} % 用于三维坐标系
\SetupExSheets{solution/print = false}
\setcounter{secnumdepth}{3} % 让三级标题有编号
\setcounter{tocdepth}{3}    % 目录显示到三级标题
\pgfplotsset{compat=1.18}

% %--------------------------------------------------------------------------
% %         General Setting
% %--------------------------------------------------------------------------

\graphicspath{{Images/}{../Images/}} %Path of figures
\setkeys{Gin}{width=0.85\textwidth} %Size of figures
\setlength{\cftbeforechapskip}{3pt} %space between items in toc
\setlength{\parindent}{0.5cm} % Idk
\input{theorems.tex} % Theorems styles and colors
\usepackage[english]{babel} %Language

\setlist[itemize]{itemsep=5pt} % Adjust the length as needed
\setlist[enumerate]{itemsep=5pt} % Adjust the length as needed



% \usepackage{lmodern} %  Latin Modern font
% \usepackage{newtxtext,newtxmath}




% %--------------------------------------------------------------------------
% %         General Informations
% %--------------------------------------------------------------------------
\newcommand{\BigTitle}{
    before learning
    }

\newcommand{\LittleTitle}{
    For you
    }
% 定义一个宏:输入三点坐标,自动计算角度并标注
\newcommand{\MarkAngleWithValue}[9]{%
  % A=(#1,#2), O=(#3,#4), B=(#5,#6),输出角度标签位置 (#7,#8),标签内容 #9
  \pgfmathsetmacro{\angA}{atan2(#2-#4,#1-#3)} % AO 的方向角
  \pgfmathsetmacro{\angB}{atan2(#6-#4,#5-#3)} % BO 的方向角
  \pgfmathsetmacro{\ang}{\angB-\angA}          % 夹角
  \ifdim \ang pt<0pt
    \pgfmathsetmacro{\ang}{\ang+360}          % 保证角度为正
  \fi
  % 画角弧
  \draw (#3+0.5,#4) arc (\angA:\angB:0.5);
  % 标注角度数值
  \node at (#7,#8) {$\ang^\circ$};
}

    
\begin{document}

% %--------------------------------------------------------------------------
% %         First pages 
% %--------------------------------------------------------------------------
\newgeometry{top=8cm,bottom=.5in,left=2cm,right=2cm}
\subfile{files/0.0.0.titlepage}
\let\cleardoublepage\clearpage
\restoregeometry
\subfile{files/0.Preface}
\let\cleardoublepage\clearpage
\subfile{files/0.zommaire}

% %--------------------------------------------------------------------------
% %         Core of the document 
% %--------------------------------------------------------------------------
% \part{Algebra}
\subfile{files/algebra}
% \subfile{files/0.0.testchap}
% \subfile{files/0.0.testchap}





% %--------------------------------------------------------------------------
% %         Bibliographie 
% %--------------------------------------------------------------------------
\nocite{*} % to cite evey things, else cite each on using : \cite{ifrs17}. 
\printbibliography %to print bibliographie

\end{document}
