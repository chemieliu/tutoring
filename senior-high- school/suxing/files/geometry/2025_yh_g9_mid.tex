\documentclass[../circle.tex]{subfiles}
\usepackage{tikz}
\usepackage{amsmath}
\begin{document}
\section{2025初中名校共同体九年级期中}
如图1,$C$、$D$ 是以 $AB$ 为直径的圆 $O$ 上的两动点,分别位于 $AB$ 两侧,且满足 $\overset{\frown}{CB} = 2 \overset{\frown}{DB}$。连接 $CD$ 交 $AB$ 于点 $E$,连接 $AC$、$BD$、$AD$。

\begin{enumerate}
    \item 证明:$AC = AE$;
    \item 若 $BD = 1$,$AD = 2$,求 $AE$ 的长;
    \item 如图2,若直径 $AB$ 为定值,当 $\triangle ABC$ 的面积最大时,求 $\triangle CEB$ 的面积与 $\triangle BED$ 的面积比。
\end{enumerate}

\vspace{1em}
\begin{center}
    \begin{minipage}{0.45\textwidth}
        \begin{tikzpicture}[scale=1]
            % 圆及直径
            \coordinate (A) at (0,0);
            \coordinate (B) at (6,0);
            \coordinate (O) at ($(A)!0.5!(B)$);
            \draw (O) circle (3cm);
            \filldraw[black] (O) circle (1pt);
            \node[left] at (A) {\small A};
            \node[right] at (B) {\small B};
            \node[below] at (O) {\small O};

            % 点C在上方圆周,点D在下方圆周
            \coordinate (C) at ($(O)+(60:3)$);
            % \coordinate (C) at (60:3); % 点B在左上
            \coordinate (D) at ($(O)+(-30:3)$);
            \node[above] at (C) {\small C};
            \node[below] at (D) {\small D};

            % 连接线段
            \draw (A) -- (B);
            \draw (C) -- (D);
            \draw (A) -- (C);
            \draw (A) -- (D);
            \draw (B) -- (D);

            % 交点E
            \coordinate (E) at (intersection of A--B and C--D);
            \filldraw[red] (E) circle (1pt);
            \node[above right] at (E) {\small E};
        \end{tikzpicture}

    \end{minipage}
    \begin{minipage}{0.45\textwidth}
        % 图形框架(图2)
        \begin{tikzpicture}[scale=1]
            % 圆及直径
            \coordinate (A2) at (0,0);
            \coordinate (B2) at (6,0);
            \coordinate (O2) at ($(A2)!0.5!(B2)$);
            \draw (O2) circle (3cm);
            \filldraw[black] (O2) circle (1pt);
            \node[left] at (A2) {\small A};
            \node[right] at (B2) {\small B};
            \node[below] at (O2) {\small O};

            % 点C在上方圆周,点D在下方圆周
            \coordinate (C2) at ($(O2)+(90:3)$);
            \coordinate (D2) at ($(O2)+(-45:3)$);
            \node[above] at (C2) {\small C};
            \node[below] at (D2) {\small D};

            % 构造E点
            \coordinate (E2) at (intersection of C2--D2 and A2--B2);
            \filldraw[red] (E2) circle (1pt);
            \node[above right] at (E2) {\small E};

            % 连接线段
            \draw (A2) -- (B2);
            \draw (A2) -- (C2);
            \draw (B2) -- (C2);
            \draw (A2) -- (D2);
            \draw (B2) -- (D2);
            \draw (C2) -- (E2) -- (B2);
            \draw (B2) -- (E2) -- (D2);
        \end{tikzpicture}
    \end{minipage}
    % 图形框架(图1)

\end{center}
\textbf{解:}
\begin{enumerate}
    \item 前两问比较简单,连接BC,设$\angle BAD = \alpha$,那么$\angle BAC = 2\alpha$ \
          $\angle ACD = 90^\circ - \alpha$,$\angle AEC = 180^\circ - \angle EAC-\angle ACE = 90^\circ - \alpha$,\newline
          第二问,用好相似或者三角函数与勾股定理
          \begin{center}
              \begin{tikzpicture}[scale=1]
                  % 圆及直径
                  \coordinate (A) at (0,0);
                  \coordinate (B) at (6,0);
                  \coordinate (O) at ($(A)!0.5!(B)$);
                  \draw (O) circle (3cm);
                  \filldraw[black] (O) circle (1pt);
                  \node[left] at (A) {\small A};
                  \node[right] at (B) {\small B};
                  \node[below] at (O) {\small O};

                  % 点C在上方圆周,点D在下方圆周
                  \coordinate (C) at ($(O)+(60:3)$);
                  % \coordinate (C) at (60:3); % 点B在左上
                  \coordinate (D) at ($(O)+(-30:3)$);
                  \node[above] at (C) {\small C};
                  \node[below] at (D) {\small D};

                  % 连接线段
                  \coordinate (F) at ($(E)!0.5!(B)$);
                  \draw (A) -- (B);
                  \draw[dashed] (C) -- (B);
                  \draw[dashed] (D) -- (F);
                  \draw (C) -- (D);
                  \draw (A) -- (C);
                  \draw (A) -- (D);
                  \draw (B) -- (D);

                  % 交点E
                  \coordinate (E) at (intersection of A--B and C--D);
                  \filldraw[red] (E) circle (1pt);
                  \node[above left] at (E) {\small E};
                  \node[above] at (F) {\small F};
              \end{tikzpicture}
          \end{center}
    \item 连接OD,OC,DF,面积之比就是高度比,即$\frac{DF}{OC}$ 。$\angle FOD = 45^\circ$,$\triangle DOA 是一个等腰三角形$,如果不会这个辅助线,直接使用高中三角函数把高度解出来也行
          \begin{center}

              \begin{minipage}{0.45\textwidth}
                  % 图形框架(图2)
                  \begin{tikzpicture}[scale=1]
                      % 圆及直径
                      \coordinate (A2) at (0,0);
                      \coordinate (B2) at (6,0);
                      \coordinate (O2) at ($(A2)!0.5!(B2)$);
                      \draw (O2) circle (3cm);
                      \filldraw[black] (O2) circle (1pt);
                      \node[left] at (A2) {\small A};
                      \node[right] at (B2) {\small B};
                      \node[below] at (O2) {\small O};

                      % 点C在上方圆周,点D在下方圆周
                      \coordinate (C2) at ($(O2)+(90:3)$);
                      \coordinate (D2) at ($(O2)+(-45:3)$);
                      \node[above] at (C2) {\small C};
                      \node[below] at (D2) {\small D};

                      % 构造E点
                      \coordinate (E2) at (intersection of C2--D2 and A2--B2);
                      \filldraw[red] (E2) circle (1pt);
                      \node[above left] at (E2) {\small E};

                      % 连接线段
                      \coordinate (F) at ($(E2)!0.5!(B2)$);
                      \draw (A2) -- (B2);
                      \draw[dashed] (C2) -- (O2);
                      \draw[dashed] (O2) -- (D2);
                      \draw[dashed] (F) -- (D2);
                      \draw (A2) -- (C2);
                      \draw (B2) -- (C2);
                      \draw (A2) -- (D2);
                      \draw (B2) -- (D2);
                      \draw (C2) -- (E2) -- (B2);
                      \draw (B2) -- (E2) -- (D2);
                      \node[above] at (F) {\small F};
                  \end{tikzpicture}
              \end{minipage}
              % 图形框架(图1)

          \end{center}

\end{enumerate}
\end{document}