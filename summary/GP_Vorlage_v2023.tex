% Header: Here are all packages used and some additional definitions
%%%%%%%%%%%%%%%%%%%%%%%%%%%%%%%%%%%%%%%%%%%%%%%%%%%%%%%%%%%%%%%%%%%

\documentclass[11pt,a4paper]{article}
\usepackage[UTF8]{ctex}  % 中文支持
\usepackage[margin=2.5cm]{geometry}
\usepackage[onehalfspacing]{setspace}
\usepackage{graphicx} % zum Einbinden von Graphiken
\usepackage[breaklinks=true,colorlinks=true,linkcolor=blue,urlcolor=blue,citecolor=blue]{hyperref} % f. Referenzen
\usepackage{amsmath,amsthm,amssymb} % Mathematik Umgebung 
\usepackage{icomma} % Intelligentes Komma, das den richtigen Abstand zwischen Dezimalzahlen als auch in Formeln wählt.
\usepackage[                ,english]{babel} % Deutsche Bezeichnungen bei Inhaltsangabe etc
\usepackage[T1]{fontenc}    % andere Schriftsatzkodierung für richtige Silbentrennung bei Umlauten
\usepackage[locale = DE,space-before-unit=true,per-mode = symbol]{siunitx} % Bessere Einheiten
\usepackage{placeins} % Definiert den Befehl “\FloatBarrier”, der die Ausgabe der davor eingebundenen Bilder erzwingt, befor der Text weiter geht. (Mit vorsicht zu verwenden)
\usepackage[natbib,abbreviate=true,doi=false,style=numeric-comp,giveninits=true,sorting=none]{biblatex} % Modernes Paket zur Erzeugung von Bibliografien (benötigt biber!)

\addbibresource{MyBibliography.bib} % Ort der .bib Datei, die die Datenbank für Literatur/Referenzen enthält.

\graphicspath{{Bilder/}}

\DeclareSIUnit{\dBm}{dBm}
\DeclareSIUnit[per-mode=reciprocal]\WN{\per\centi\meter}

%%%%%%%%%%%%%%%%%%%%%%%%%%%%%%%%%%%%%%%%%%%%%%%%%%%%%%%%%%%%%%%%%%%
\begin{document}
%
\title{2025工作总结}
\author{Liu}
\date{\today}
\maketitle
\vfill
\renewcommand\abstractname{Abstract}
\section*{\abstractname}
\textit{2025年已过,回顾2025年,展望2026年。总结2025成功、失败的经验与教训,为2026更进一步而努力。做一个对家庭充满爱意、对生活充满热情、对事业充满激情的人。}\\
\thispagestyle{empty}
%
%
\tableofcontents
\thispagestyle{empty}
\cleardoublepage
\pagenumbering{arabic}
\newpage
%
%
\section{Introduction}
\label{sec:Einleitung}
%
\textit{2025年是我人生重要的一年,这一年是我家庭、事业、人生态度发生改变的一年、是我人生迈上新台阶的第一年。在2026年第一天,充分总结2025年、积极展望2026年。本着对天地的敬畏之心、对生活的积极态度、对未来的乐观展望,郑重写下本文,并以期形成常态化,每年都对上一年进行总结与展望。}\\

%
\section{2025年大事}
\label{sec:Grundlagen}
%
\textit{主要记录2025年我人生与生活中的一些大事、心态变化、学习进展等。}\\
\subsection{结婚}
2025年,我与我的爱人在山东老家举办婚礼。在家人、亲朋好友的帮助下,我们的婚礼成功举办。结婚意味着我们有了一个小家庭,为小家庭努力奋斗,给我带来了无穷的动力。
我坚信我们小家在杭州会越来越好!个人私生活部分不会过多介绍,这是我的性格决定的。
\subsection{买车}
买车其实在结婚之前,前面也有介绍,这辆车是广汽本田皓影插混。这辆车我前面多次发文介绍、并坦言她是我们小家的一个重要成员。这辆车是我与父亲亲自接回家、并按照山东习俗给她举行了一个庄重的仪式。
我驾驶她把我老婆从娘家接到我家、她全程参与我的婚礼、并载着我与老婆从山东老家平安到达杭州。她陪我走过杭州大街小巷、助力我事业发展。我会珍惜我人生的第一辆车。
\subsection{出国}
我与我老婆第一次出国,第一站选择了新加坡。
\subsection{MacBook}
选择macbook pro是我本年做出的一个重要选择。现在看来这个选择无比正确,选择能够提升工作效率、提升工作质量的工具是十分必要的。
有些东西苹果做的确实好,比如显示屏、操作系统、生态系统等。我本人还需要写代码,MacBook对我工作帮助很大。
\subsection{iPhone }
基于macbook pro的良好表现,我给我老婆也选择了苹果手机。不基于情绪、而是基于理性选择工具,这也是我2025年人生态度的一个改变。




%
\section{工作总结}
\label{sec:ExpAufb}
%
\textit{2025年比2024年要好很多,我自身也提升很多。2025年的进展让我对2026年更有信心。}\\
\subsection{小初高全覆盖}
2025年,我实现了从小学、初中、高中的全覆盖教学。中考数学仍然是2025年的重点、奥数权重降低、高中权重适当提升。
成功组建了几个小组教学、多年级混班教学、线上线下结合教学等多种形式。
\subsection{不搞预付费}
\section{Ergebnisse}
\label{sec:Ergebnisse}
%
\textit{Was ist(sind) die gemessene Antwort(en) auf die Hauptfrage(n)?}\\
Abbildung \ref{fig:examplfig} zeigt ein Beispiel für eine Abbildung in Haupttext.
\begin{figure}[htb!]
  \centering
  \includegraphics[width=0.65\textwidth]{good_example_plot}
  \caption{\label{fig:examplfig}Ein typischer Graph in einem Bericht. Im Bildunterschrift werden die wesentliche Informationen über den Graph gegeben.
  }
\end{figure}
%
\section{Schlussfolgerungen}
\label{sec:Schlussf}
%
\textit{Was ist die Endantwort und soll ihr vertraut werden?  Wie hätte man den Versuch anders oder besser durchführen können?}\\
Die Referenz~\cite{GP1StromSpannung} soll ein Beispiel sein, wie man ein Praktikums"=Skript zitieren sollte.
%
\input{Anhang} % Der Anhang ist in einer externen Datei "Anhang.tex" und wird hier in das Dokument eingefügt.

\printbibliography[]
\vfill
\section*{Erklärung}

Hiermit versichern wir, dass der vorliegende Bericht selbständig verfasst wurde und alle notwendigen Quellen und Referenzen angegeben sind.

\begin{tabular}{@{}p{2.5in}p{2.5in}@{}}
  \\[5\bigskipamount]
  \dotfill  & \dotfill \\
  Student 1 & Date     \\[5\bigskipamount]
  \dotfill  & \dotfill \\
  Student 2 & Date     \\
  \centering
\end{tabular}

\end{document}
