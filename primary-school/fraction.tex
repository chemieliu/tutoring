\documentclass[a4paper,12pt]{article}
\usepackage[UTF8]{ctex}  % 中文支持

%%%%%%%%%%%%%%%%%%%%%
% pour écrire avec les accents directement dans le texte

\usepackage[frenchb,english]{babel}

 \usepackage[T1]{fontenc}
 \usepackage[colorlinks=false, urlcolor=black, breaklinks, pagebackref, citebordercolor={0 0 0}, filebordercolor={0 0 0}, linkbordercolor={0 0 0}, pagebordercolor={0 0 0}, runbordercolor={0 0 0}, urlbordercolor={0 0 0}, pdfborder={0 0 0}]{hyperref} 
%%%%%%%%%%%%%%%%%%%%% 

\usepackage{float}
\usepackage{amsmath}
\usepackage{amsfonts}
\usepackage{amssymb}
\usepackage{geometry}
\usepackage{graphicx}
\usepackage{hyperref}
\usepackage{icomma}
\usepackage{latexsym}
\usepackage[numbers]{natbib}
\usepackage{textcomp}
\usepackage[all]{xy}
\usepackage{cancel}
\usepackage{color}
\usepackage{dsfont}
\usepackage{subfig}
\usepackage{caption}
\usepackage{dsfont}
\usepackage{extarrows}
\usepackage{shorttoc}
\usepackage{listings}
\usepackage[utf8]{inputenc}
\usepackage{lmodern}
\usepackage{xcolor}
\usepackage{moreverb}
\usepackage{amsmath, amsfonts, amssymb} % Ecuaciones, caracteres y símbolos especiales
\usepackage{hyperref, url}  % Links y Hyperlinks en el documento
\usepackage{fancyhdr}
\usepackage{tikz}
\usetikzlibrary{decorations.pathreplacing, calc} % 加载必要的库
\lstset{language=scilab}

\lstset{
        basicstyle=\ttfamily,
        emphstyle=\color{blue},
        columns=flexible,
        breaklines=true,
        keywordstyle=\color{blue},
        commentstyle=\color{red},
}

   \newcommand{\bigO}[1]{\ensuremath{\mathop{}\mathopen{}O\mathopen{}\left(#1\right)}}
   \newcommand{\smallO}[1]{\ensuremath{\mathop{}\mathopen{}o\mathopen{}\left(#1\right)}}

\captionsetup{font=footnotesize}

\setcounter{tocdepth}{3}

\makeatletter
\def\clap#1{\hbox to 0pt{\hss #1\hss}}%
\def\ligne#1{%
\hbox to \hsize{%
\vbox{\centering #1}}}%
\def\haut#1#2#3{%
\hbox to \hsize{%
\rlap{\vtop{\raggedright #1}}%
\hss
\clap{\vtop{\centering #2}}%
\hss
\llap{\vtop{\raggedleft #3}}}}%
\def\bas#1#2#3{%
\hbox to \hsize{%
\rlap{\vbox{\raggedright #1}}%
\hss
\clap{\vbox{\centering #2}}%
\hss
\llap{\vbox{\raggedleft #3}}}}%
\def\maketitle{%
\thispagestyle{empty}\vbox to \vsize{%
\haut{}{\@blurb}{}
\vfill
\vspace{0.5cm}
\hrule height 2pt
\vspace{0.2cm}
\par
\begin{center}
\fontseries{bx}
\huge \@title
\end{center}
\vspace{0.2cm}
\par
\hrule height 2pt
\par
\vspace{2cm}
% \begin{center}
% \Large
% % \textit{Auteurs:}
% \end{center}
\begin{center}
\Large 
\@author
\par
\end{center}
\vspace{1cm}
% \begin{center}
% \Large
% \textit{Responsable:}
% \end{center}
% \begin{center}
% \Large 
% Christian Robert
% \par
% \end{center}
% \vfill
\vfill
\bas{}{\begin{flushright}
\large
\@location
\end{flushright}}{}
}%
\cleardoublepage
}
\def\date#1{\def\@date{#1}}
\def\author#1{\def\@author{#1}}
\def\title#1{\def\@title{#1}}
\def\location#1{\def\@location{#1}}
\def\blurb#1{\def\@blurb{#1}}
\date{\today}
\author{}
\title{}
\location{Lyon}\blurb{}
\makeatother
\title{三年级分数应用题}
\author{Liu}
\location{Hangzhou}
\blurb{%
\begin{center}
% \fbox{\includegraphics[width=6em]{Images/isfa.jpg}}
\end{center}
%Pour\\
%\textbf{type de stage}\\[1em]
%Maître de stage : moralles\\
}%

\begin{document}

\maketitle

\tableofcontents

\newpage

\part*{Introduction}\label{sec:name}
\addcontentsline{toc}{part}{Introduction}

\textcolor{blue}{分数应用题简介}

\vspace{0.5cm}
\hspace{1.5cm}
这里更多是理清部分与整体、比例关系、单位“1”等关系,对分数定义中的平均分有个更好的了解。




\section{直接应用乘除做题}

不用考虑混合运算,直接应用乘除做题。

\subsection{总体到部分}


\noindent 小明有36个苹果,卖掉其中的$\frac{1}{3}$。卖掉多少个苹果? \newline
\textbf{解:}
\begin{tikzpicture}[scale=0.8, >=stealth]
        % 定义两个点
        \coordinate (P1) at (0,0);
        \coordinate (P2) at (3,0);
        \coordinate (P3) at (0,1);
        \coordinate (P4) at (3,1);
        \coordinate (P5) at (9,0);
        \coordinate (P6) at (6,0);
        % 1. 在起点 (0,0) 处放置文字 (使用 'left' 定位)
        % 首先定义起点 (0,0) 为一个节点 N1,并放置文字
        \node (label1) at (P1) [left] {总共};
        \node (label2) at (P3) [left] {$\frac{1}{3}$};
        % 然后从这个点开始画线段
        \draw[ thick] (P1) -- (P2);
        \draw[ thick] (P3) -- (P4);
        \draw[thick,red] (P1) -- (P5);
        % \draw[decorate, decoration={brace, amplitude=7pt, mirror}] (P1) -- (P2)
        % node[midway, below=8pt] {35}; % mirror 选项使大括号朝向P1和P2定义的路径的内部
        \draw[decorate, decoration={brace, amplitude=5pt}] (P3) -- (P4)
        node[midway, above=6pt] {?};

        \draw[dashed] (P1) -- ++(0, 0.5);
        \draw[dashed] (3,0) -- ++(0, 0.5);
        \draw[<->,thick] (0,0.3) -- (9,0.3) node[midway, above] { $36$};
        \draw[dashed] (P1) -- ++(0, 0.5);
        \draw[dashed] (P6) -- ++(0, 0.5);
        \draw[dashed] (P5) -- ++(0, 0.5);

        \draw[decorate, decoration={brace, amplitude=10pt, mirror}] (P1) -- (P2)
        node[midway, below=8pt] {?}; % mirror 选项使大括号朝向P1和P2定义的路径的内部

\end{tikzpicture}

\vspace{0.5cm}
看到$\frac{1}{3}$,就想到平均分成三份,这里是用除法。\newline
$36 \div 3 = 12$\newline
\textbf{答:}卖掉12个苹果。\newline

\subsection{部分到整体}


\noindent 有一些苹果,小明卖掉其中的$\frac{1}{3}$,一共卖掉9个。原来一共有多少个苹果? \newline
\textbf{解:}
\begin{tikzpicture}[scale=0.8, >=stealth]
        % 定义两个点
        \coordinate (P1) at (0,0);
        \coordinate (P2) at (3,0);
        \coordinate (P3) at (0,1);
        \coordinate (P4) at (3,1);
        \coordinate (P5) at (9,0);
        \coordinate (P6) at (6,0);
        % 1. 在起点 (0,0) 处放置文字 (使用 'left' 定位)
        % 首先定义起点 (0,0) 为一个节点 N1,并放置文字
        \node (label1) at (P1) [left] {总共};
        \node (label2) at (P3) [left] {$\frac{1}{3}$};
        % 然后从这个点开始画线段
        \draw[ thick] (P1) -- (P2);
        \draw[ thick] (P3) -- (P4);
        \draw[thick,red] (P1) -- (P5);
        % \draw[decorate, decoration={brace, amplitude=7pt, mirror}] (P1) -- (P2)
        % node[midway, below=8pt] {35}; % mirror 选项使大括号朝向P1和P2定义的路径的内部
        \draw[decorate, decoration={brace, amplitude=5pt}] (P3) -- (P4)
        node[midway, above=6pt] {9};

        \draw[dashed] (P1) -- ++(0, 0.5);
        \draw[dashed] (3,0) -- ++(0, 0.5);
        \draw[<->,thick] (0,0.3) -- (3,0.3) node[midway, above] { $9$};
        \draw[dashed] (P1) -- ++(0, 0.5);
        \draw[dashed] (P6) -- ++(0, 0.5);
        \draw[dashed] (P5) -- ++(0, 0.5);

        \draw[decorate, decoration={brace, amplitude=10pt, mirror}] (P1) -- (P5)
        node[midway, below=8pt] {?}; % mirror 选项使大括号朝向P1和P2定义的路径的内部

\end{tikzpicture}

\vspace{0.5cm}
看到$\frac{1}{3}$,就想到平均分成三份,这里是用乘法。\newline
整体分成了三份,每份是9个苹果。\newline
$9 \times 3 = 27$\newline
\textbf{答:}一共有27个苹果。\newline







\section{不直接应用乘除做题}

基本上要把整体分成几部分,得到每部分占比是多少,再进行计算。

\subsection{总体到部分}


\noindent 小明有36个苹果,卖掉其中的$\frac{3}{4}$。还剩下多少个苹果? \newline
\textbf{解:}
\begin{tikzpicture}[scale=0.8, >=stealth]
        % 定义两个点
        \coordinate (P1) at (0,0);
        \coordinate (P2) at (3,0);
        \coordinate (P3) at (0,1);
        \coordinate (P4) at (3,1);
        \coordinate (P5) at (9,0);
        \coordinate (P7) at (12,0);
        \coordinate (P6) at (6,0);
        % 1. 在起点 (0,0) 处放置文字 (使用 'left' 定位)
        % 首先定义起点 (0,0) 为一个节点 N1,并放置文字
        \node (label1) at (P1) [left] {总共};

        % 然后从这个点开始画线段
        \draw[ thick] (P1) -- (P2);

        \draw[thick,red] (P1) -- (P7);
        % \draw[decorate, decoration={brace, amplitude=7pt, mirror}] (P1) -- (P2)
        % node[midway, below=8pt] {35}; % mirror 选项使大括号朝向P1和P2定义的路径的内部


        \draw[dashed] (P1) -- ++(0, 0.5);
        \draw[dashed] (3,0) -- ++(0, 0.5);
        \draw[<->,thick] (0,0.3) -- (12,0.3) node[midway, above] { $36$};
        \draw[dashed] (P1) -- ++(0, 0.5);
        \draw[dashed] (P6) -- ++(0, 0.5);
        \draw[dashed] (P5) -- ++(0, 0.5);
        \draw[dashed] (P7) -- ++(0, 0.5);

        \draw[decorate, decoration={brace, amplitude=10pt, mirror}] (P1) -- (P5)
        node[midway, below=8pt] {$\frac{3}{4}$};
        \draw[decorate, decoration={brace, amplitude=10pt, mirror}] (P5) -- (P7)
        node[midway, below=8pt] {?};  % mirror 选项使大括号朝向P1和P2定义的路径的内部

\end{tikzpicture}

\vspace{0.5cm}
总共=卖掉+剩余 \newline
看到一共36个,$\frac{3}{4}$,有了总体和份数,就想到平均分成四份,可以得到每份是多少。\newline
每份:$36 \div 4 = 9$\newline
两个思路,先求卖掉是多少,再用总体减去卖掉的;
卖掉三份,剩余一份。\newline
卖掉:$9 \times 3 = 27$\newline
剩余:$36 - 27 = 9$\newline
\textbf{或者}\newline
剩余:
$4-3=1$\newline
$9 \times 1 = 9$\newline




\textbf{答:}还剩下9个苹果。\newline

\subsection{部分到整体}


\noindent 有一些苹果,小明卖掉其中的$\frac{2}{5}$,还剩下9个。原来一共有多少个苹果? \newline
\textbf{解:}
\begin{tikzpicture}[scale=0.8, >=stealth]
        % 定义两个点
        \coordinate (P1) at (0,0);
        \coordinate (P2) at (3,0);
        \coordinate (P3) at (0,1);
        \coordinate (P4) at (3,1);
        \coordinate (P5) at (9,0);
        \coordinate (P7) at (12,0);
        \coordinate (P8) at (15,0);
        \coordinate (P6) at (6,0);
        % 1. 在起点 (0,0) 处放置文字 (使用 'left' 定位)
        % 首先定义起点 (0,0) 为一个节点 N1,并放置文字
        \node (label1) at (P1) [left] {总共};

        \draw[ thick] (P1) -- (P2);

        \draw[thick,red] (P1) -- (P8);
        % \draw[decorate, decoration={brace, amplitude=7pt, mirror}] (P1) -- (P2)
        % node[midway, below=8pt] {35}; % mirror 选项使大括号朝向P1和P2定义的路径的内部


        \draw[dashed] (P1) -- ++(0, 0.5);
        \draw[dashed] (3,0) -- ++(0, 0.5);
        %     \draw[<->,thick] (0,0.3) -- (3,0.3) node[midway, above] { $9$};
        \draw[decorate, decoration={brace, amplitude=10pt, mirror}] (P6) -- (P1)
        node[midway, above=8pt] { $\frac{2}{5}$};
        \draw[decorate, decoration={brace, amplitude=10pt, mirror}] (P8) -- (P6)
        node[midway, above=8pt] { 9};
        \draw[dashed] (P1) -- ++(0, 0.5);
        \draw[dashed] (P6) -- ++(0, 0.5);
        \draw[dashed] (P5) -- ++(0, 0.5);
        \draw[dashed] (P7) -- ++(0, 0.5);
        \draw[dashed] (P8) -- ++(0, 0.5);

        \draw[decorate, decoration={brace, amplitude=10pt, mirror}] (P1) -- (P8)
        node[midway, below=8pt] {?}; % mirror 选项使大括号朝向P1和P2定义的路径的内部

\end{tikzpicture}

\vspace{0.5cm}
看到$\frac{2}{5}$,就想到平均分成5份\newline
总共=卖掉+剩余 \newline
卖掉$\frac{2}{5}$,就是卖掉2份,剩余3份\newline
剩余:$5-2=3$\newline
每份:$9 \div 3 = 3$\newline
整体:$3 \times 5 = 15$\newline

\textbf{答:}一共有15个苹果。\newline

\section{综合应用做题}
由于时间关系,我不先不做图,直接文字解释。
\subsection{多个部分之和}
一本书共100页,小明上午读了全书的$\frac{2}{5}$,下午读了全书的$\frac{1}{4}$,他一共读了多少页?\newline
\textbf{解:}\newline
这里都是在总体中进行份数的计算,然后把部分加起来。\newline

上午读:$100 \div 5 \times 2 = 40$\newline
下午读:$100 \div 4 \times 1 = 25$\newline
一共读:$40 + 25 = 65$\newline
\textbf{答:}一共读了65页。\newline
一本书共100页,小明上午读了全书的$\frac{2}{5}$,下午读了剩下部分的$\frac{1}{4}$,他一共读了多少页?\newline
\textbf{解:}\newline
下午的部分是剩下部分,所以要减去已经读的。再求每份是多少。\newline
上午读:$100 \div 5 \times 2 = 40$\newline
剩下部分:$100 - 40 = 60$\newline
下午读:$60 \div 4 \times 1 = 15$\newline
一共读:$40 + 15 = 55$\newline
\textbf{答:}一共读了55页。\newline






\newpage

\part*{Conclusion}\label{sec:name}
\addcontentsline{toc}{part}{Conclusion}
分数应用题,主要是理清部分与整体、先后步骤。
这里需要对每个步骤的整体、部分有个了解。
更多是一种细致认真,理清各项数据。\newline
如读书那个题目,记录这本书一共多少页,上午看了多少,还剩多少。
上午剩下的就是下午的总体,然后把每次阅读的页数相加。\newline
这里的计算不算难,要明白这里的“1”,它不是数字“1”,而是“一个整体”的概念单位。"unit one / the whole"转成数学语言就是"1"。
把整体看作单位“1”,本质上是将复杂的、具体的数量抽象化。它让我们不再纠结于具体的数字(比如 500克或 100人),而是专注于部分与整体之间的比例关系。\newline
建议用好草稿纸,用文字标记一下每个数字代表什么,就跟我上面写的一样。
\newpage



\end{document}