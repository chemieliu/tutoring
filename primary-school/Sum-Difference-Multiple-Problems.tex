% Plantilla simple para tareas de la Licenciatura en Física
% Fer Flores - Universidad de Guadalajara - Noviembre 2023

%%%%%%%%%%%%%%%%%%%%%%%%%%%%%%%%%%%%%%%%%%-PREÁMBULO-%%%%%%%%%%%%%%%%%%%%%%%%%%%%%%%%%%%%%%%%%%

% Paqueterías

\documentclass{assignment}
\usepackage[UTF8]{ctex}  % 中文支持
\usepackage[pdftex]{graphicx} % FIGURAS
\usepackage{xcolor}
\definecolor{LightGray}{gray}{0.95}
\usepackage{fancyvrb, minted} % CÓDIGO
\usepackage[letterpaper, margin = 2.5cm]{geometry} % TAMAÑO DE PÁGINA Y MÁRGENES
\usepackage[T1]{fontenc} % Importante para acentos automáticos y símbolos de escritura
\usepackage[english]{babel} % Importante para Español
\usepackage{amsmath, amsfonts, amssymb} % Ecuaciones, caracteres y símbolos especiales
\usepackage{hyperref, url}  % Links y Hyperlinks en el documento
\usepackage{fancyhdr}
\usepackage{tikz}
\usepackage{titlesec}
\usetikzlibrary{decorations.pathreplacing, calc} % 加载必要的库

%-----------------------------------------ETIQUETAS--------------------------------------------

\student{Liu}                             % NOMBRE
\semester{Grade3}                                % SEMESTRE (202X A/B)
\date{\today}                                   % Fecha (Modifica a DD/MM/AAAA)

\courselabel{ONLINE}          % CLAVE Y MATERIA
\exercisesheet{和差倍}{从基础到高级}     % NÚMERO Y TÍTULO DE LA TAREA

\school{Yuhang}          % CARERA (Física, la mejor carrera)
\university{Mr. Liu's class}         % LA PODEROSÍSIMA

%%%%%%%%%%%%%%%%%%%%%%%%%%%%%%%%%%%%%%%%%%-DOCUMENTO-%%%%%%%%%%%%%%%%%%%%%%%%%%%%%%%%%%%%%%%%%%%%

\begin{document}

%-----------------------------------------------------------------------------------------------
\begin{problem}

\section{简介}

\noindent 这类问题通常围绕两个或多个未知量,通过它们的“和”、“差”或“倍数关系”来求解这些未知量。 \newline
For example:The sum of two numbers is 36, and their difference is 12. What are the numbers? \newline


\section{解决方法}
\noindent 高年级与低年级的解题思路是不同的,本质上讲这两种方法是等价的。\newline 
突破口在找到\textbf{等量关系}。
\begin{enumerate}
    \item 代数法(高年级)

    \item 线段图法(低年级)


\end{enumerate}

\subsection{基础题型}

\noindent 只需要应用简单加、减、乘或除运算即可

\subsubsection{和}

\noindent 小明有36个苹果,小红比小明多12个苹果。小红有多少个苹果? \newline
\textbf{解:}
\begin{tikzpicture}[scale=0.8, >=stealth]
    % 定义两个点
    \coordinate (P1) at (0,0);
    \coordinate (P2) at (3,0);
    \coordinate (P3) at (0,1);
    \coordinate (P4) at (3,1);
    \coordinate (P5) at (4,0);
    % 1. 在起点 (0,0) 处放置文字 (使用 'left' 定位)
    % 首先定义起点 (0,0) 为一个节点 N1,并放置文字
    \node (label1) at (0, 0) [left] {小红};
    \node (label2) at (P3) [left] {小明};
    % 然后从这个点开始画线段
    \draw[ thick] (P1) -- (P2);
    \draw[ thick] (P3) -- (P4);
    \draw[dashed ,thick,red] (P2) -- (P5);
    % \draw[decorate, decoration={brace, amplitude=7pt, mirror}] (P1) -- (P2)
    % node[midway, below=8pt] {35}; % mirror 选项使大括号朝向P1和P2定义的路径的内部
    \draw[decorate, decoration={brace, amplitude=5pt}] (P3) -- (P4)
    node[midway, above=6pt] {36};
    \draw[dashed] (0,0) -- ++(0, 0.5);
    \draw[dashed] (3,0) -- ++(0, 0.5);
    \draw[<->,thick] (0,0.3) -- (3,0.3) node[midway, above] { $36$};
    \draw[dashed] (3,0) -- ++(0, 0.5);
    \draw[dashed] (4,0) -- ++(0, 0.5);
    \draw[<->,thick] (3,0.3) -- (4,0.3) node[midway, above] { $12$};
    \draw[decorate, decoration={brace, amplitude=10pt, mirror}] (P1) -- (P5)
    node[midway, below=8pt] {?}; % mirror 选项使大括号朝向P1和P2定义的路径的内部

\end{tikzpicture}

\vspace{0.5cm}
36 + 12 = 48 \newline

答:小红有48个苹果。\newline
\subsubsection{差}

\noindent 小明有36个苹果,小红比小明少12个苹果。小红有多少个苹果? \newline
\textbf{解:}
\begin{tikzpicture}[scale=0.8, >=stealth]
    % 定义两个点
    \coordinate (P1) at (0,0);
    \coordinate (P2) at (3,0);
    \coordinate (P3) at (0,1);
    \coordinate (P4) at (3,1);
    \coordinate (P5) at (2,0);
    % 1. 在起点 (0,0) 处放置文字 (使用 'left' 定位)
    % 首先定义起点 (0,0) 为一个节点 N1,并放置文字
    \node (label1) at (P1) [left] {小红};
    \node (label2) at (P3) [left] {小明};
    % 然后从这个点开始画线段
    \draw[ thick] (P1) -- (P5);
    \draw[ thick] (P3) -- (P4);
    \draw[dashed ,thick,red] (P2) -- (P5);
    % \draw[decorate, decoration={brace, amplitude=7pt, mirror}] (P1) -- (P2)
    % node[midway, below=8pt] {35}; % mirror 选项使大括号朝向P1和P2定义的路径的内部
    \draw[decorate, decoration={brace, amplitude=5pt}] (P3) -- (P4)
    node[midway, above=6pt] {36};
    \draw[dashed] (P1) -- ++(0, 0.5);
    \draw[dashed] (3,0) -- ++(0, 0.5);
    \draw[<->,thick] (0,0.3) -- (3,0.3) node[midway, above] { $36$};
    \draw[dashed] (3,0) -- ++(0, -0.5);
    \draw[dashed] (2,0) -- ++(0, -0.5);
    \draw[<->,thick] (2,-0.3) -- (3,-0.3) node[midway, below] { $12$};
    \draw[decorate, decoration={brace, amplitude=10pt, mirror}] (P1) -- (P5)
    node[midway, below=8pt] {?}; % mirror 选项使大括号朝向P1和P2定义的路径的内部

\end{tikzpicture}

\vspace{0.5cm}
36 - 12 = 24 \newline

答:小红有24个苹果。\newline

\subsubsection{倍}

\noindent \textbf{\textcolor{red}{乘:}}小明有36个苹果,小红苹果数量是小明的3倍。小红有多少个苹果? \newline
\textbf{解:}
\begin{tikzpicture}[scale=0.8, >=stealth]
    % 定义两个点
    \coordinate (P1) at (0,0);
    \coordinate (P2) at (3,0);
    \coordinate (P3) at (0,1);
    \coordinate (P4) at (3,1);
    \coordinate (P5) at (9,0);
    \coordinate (P6) at (6,0);
    % 1. 在起点 (0,0) 处放置文字 (使用 'left' 定位)
    % 首先定义起点 (0,0) 为一个节点 N1,并放置文字
    \node (label1) at (P1) [left] {小红};
    \node (label2) at (P3) [left] {小明};
    % 然后从这个点开始画线段
    \draw[ thick] (P1) -- (P2);
    \draw[ thick] (P3) -- (P4);
    \draw[thick,red] (P1) -- (P5);
    % \draw[decorate, decoration={brace, amplitude=7pt, mirror}] (P1) -- (P2)
    % node[midway, below=8pt] {35}; % mirror 选项使大括号朝向P1和P2定义的路径的内部
    \draw[decorate, decoration={brace, amplitude=5pt}] (P3) -- (P4)
    node[midway, above=6pt] {36};

    \draw[dashed] (P1) -- ++(0, 0.5);
    \draw[dashed] (3,0) -- ++(0, 0.5);
    \draw[<->,thick] (0,0.3) -- (3,0.3) node[midway, above] { $36$};
    \draw[dashed] (P1) -- ++(0, 0.5);
    \draw[dashed] (P6) -- ++(0, 0.5);
    \draw[dashed] (P5) -- ++(0, 0.5);
    \draw[<->,thick] (0,0.3) -- (3,0.3) node[midway, above] { $36$};
    \draw[<->,thick] (3,0.3) -- (6,0.3) node[midway, above] { $36$};
    \draw[<->,thick] (6,0.3) -- (9,0.3) node[midway, above] { $36$};
    \draw[decorate, decoration={brace, amplitude=10pt, mirror}] (P1) -- (P5)
    node[midway, below=8pt] {?}; % mirror 选项使大括号朝向P1和P2定义的路径的内部

\end{tikzpicture}

\vspace{0.5cm}
$36\times3$  = 108 \newline

答:小红有108个苹果。\newline
\newline
\noindent \textbf{\textcolor{red}{除:}} 小明有36个苹果,小明苹果数量是小红的3倍。小红有多少个苹果? \newline
\textbf{解:}
\begin{tikzpicture}[scale=0.8, >=stealth]
    % 定义两个点
    \coordinate (P1) at (0,0);
    \coordinate (P2) at (3,0);
    \coordinate (P3) at (0,1);
    \coordinate (P4) at (3,1);
    \coordinate (P5) at (1,0);
    \coordinate (P6) at (2,0);
    % 1. 在起点 (0,0) 处放置文字 (使用 'left' 定位)
    % 首先定义起点 (0,0) 为一个节点 N1,并放置文字
    \node (label1) at (P1) [left] {小红};
    \node (label2) at (P3) [left] {小明};
    % 然后从这个点开始画线段
    \draw[ thick] (P1) -- (P2);
    \draw[ thick] (P3) -- (P4);
    \draw[thick,red] (P1) -- (P5);
    % \draw[decorate, decoration={brace, amplitude=7pt, mirror}] (P1) -- (P2)
    % node[midway, below=8pt] {35}; % mirror 选项使大括号朝向P1和P2定义的路径的内部
    \draw[decorate, decoration={brace, amplitude=5pt}] (P3) -- (P4)
    node[midway, above=6pt] {36};

    \draw[dashed] (P1) -- ++(0, 0.5);
    \draw[dashed] (3,0) -- ++(0, 0.5);
    \draw[<->,thick] (0,0.3) -- (3,0.3) node[midway, above] { $36$};
    \draw[dashed] (P1) -- ++(0, 0.5);
    \draw[dashed] (P6) -- ++(0, 0.2);
    \draw[dashed] (P5) -- ++(0, 0.2);
    \draw[<->,thick] (0,0.3) -- (3,0.3) node[midway, above] { $36$};

    \draw[decorate, decoration={brace, amplitude=10pt, mirror}] (P1) -- (P5)
    node[midway, below=8pt] {?}; % mirror 选项使大括号朝向P1和P2定义的路径的内部

\end{tikzpicture}

\vspace{0.5cm}
$36\div3$  = 12 \newline

答:小红有12个苹果。\newline

\noindent 注意:倍数关系中,乘法和除法是互为逆运算的。\newline
% \vspace{0.5cm}
\begin{enumerate}
    \item 认真读题

    \item 认真计算

    \item 认真作答
\end{enumerate}
\subsection{提升题型}
\noindent 这里可以使用代数法或者线段图法来解决问题,四五年级以后推荐使用代数法。
\subsubsection{和倍}
\begin{itemize}
    \item Problem 1
          \newline
          \noindent 小明和小红一共有36个苹果,小红苹果数量是小明的3倍。小红有多少个苹果? \newline
          \textbf{解:}
          \begin{tikzpicture}[scale=0.8, >=stealth]
              % 定义两个点
              \coordinate (P1) at (0,0);
              \coordinate (P2) at (3,0);
              \coordinate (P3) at (0,1);
              \coordinate (P4) at (3,1);
              \coordinate (P5) at (9,0);
              \coordinate (P6) at (6,0);
              % 1. 在起点 (0,0) 处放置文字 (使用 'left' 定位)
              % 首先定义起点 (0,0) 为一个节点 N1,并放置文字
              \node (label1) at (P1) [left] {小红};
              \node (label2) at (P3) [left] {小明};
              \node (label2) at (0,-1) [left] {小明+小红};
              % 然后从这个点开始画线段
              
              \draw[ thick] (P3) -- (P4);
              \draw[thick,red] (P1) -- (P5);
        
              \draw[thick] (0,-1) -- (3,-1);
              \draw[thick,red] (3,-1) -- (12,-1);
              % \draw[decorate, decoration={brace, amplitude=7pt, mirror}] (P1) -- (P2)
              % node[midway, below=8pt] {35}; % mirror 选项使大括号朝向P1和P2定义的路径的内部
              \draw[decorate, decoration={brace, amplitude=5pt}] (P3) -- (P4)
              node[midway, above=6pt] {x};
              \draw[decorate, decoration={brace, amplitude=5pt}] (9.5,2) -- (9.5,0)
              node[midway, right=6pt] {36};

              \draw[dashed] (P1) -- ++(0, 0.5);
              \draw[dashed] (3,0) -- ++(0, 0.5);

              \draw[dashed] (P1) -- ++(0, 0.5);
              \draw[dashed] (P6) -- ++(0, 0.5);
              \draw[dashed] (P5) -- ++(0, 0.5);
              % \draw[<->,thick] (0,0.3) -- (3,0.3) node[midway, above] { ?};
              % \draw[<->,thick] (3,0.3) -- (6,0.3) node[midway, above] { ?};
              % \draw[<->,thick] (6,0.3) -- (9,0.3) node[midway, above] { ?};
              \draw[dashed] (0,-1) -- ++(0, 0.5);
              \draw[dashed] (3,-1) -- ++(0, 0.5);
              \draw[dashed] (6,-1) -- ++(0, 0.5);
              \draw[dashed] (9,-1) -- ++(0, 0.5);
              \draw[dashed] (12,-1) -- ++(0, 0.5);
              \draw[decorate, decoration={brace, amplitude=10pt}] (12,-1) -- (0,-1)
              node[midway, below=8pt] {36};

          \end{tikzpicture}

          \vspace{0.5cm}

          $3$ + 1 = 4 \newline
          \indent $36\div4$  = 9(个) \newline
          \indent $3\times9$  = 27(个) \newline

          答:小红有27个苹果。\newline
          \textbf{列方程:} \newline
          \indent 设小明有x个苹果,则小红有3x个苹果。\newline
          \indent 根据题意可得:x + 3x = 36 \newline
          \indent 4x = 36 \newline
          \indent x = 9 \newline
          \indent 小红有3x = 27个苹果。\newline

    \item Problem 2
          \newline
          \noindent 小明和小红一共有30个苹果,小红苹果数量比小明的3倍少两个。小红有多少个苹果? \newline
          \textbf{解:}
          \begin{tikzpicture}[scale=0.8, >=stealth]
              % 定义两个点
              \coordinate (P1) at (0,0);
              \coordinate (P2) at (3,0);
              \coordinate (P3) at (0,1);
              \coordinate (P4) at (3,1);
              \coordinate (P5) at (9,0);
              \coordinate (P7) at (8,0);
              \coordinate (P6) at (6,0);
              % 1. 在起点 (0,0) 处放置文字 (使用 'left' 定位)
              % 首先定义起点 (0,0) 为一个节点 N1,并放置文字
              \node (label1) at (P1) [left] {小红};
              \node (label2) at (P3) [left] {小明};
              \node (label2) at (0,-1) [left] {小明+小红};
              % 然后从这个点开始画线段
              \draw[ thick] (P1) -- (P2);
              \draw[ thick] (P3) -- (P4);
              \draw[dashed, thick] (P5) -- (P7);
              \draw[thick,red] (P1) -- (P7);
              \draw[thick,red] (0,-1) -- (11,-1);
              \draw[thick,dashed] (11,-1) -- (12,-1);
              % \draw[decorate, decoration={brace, amplitude=7pt, mirror}] (P1) -- (P2)
              % node[midway, below=8pt] {35}; % mirror 选项使大括号朝向P1和P2定义的路径的内部
              \draw[decorate, decoration={brace, amplitude=5pt}] (P3) -- (P4)
              node[midway, above=6pt] {$x$};
              \draw[decorate, decoration={brace, amplitude=5pt}] (9.5,2) -- (9.5,0)
              node[midway, right=6pt] {30};

              \draw[dashed] (P1) -- ++(0, 0.5);
              \draw[dashed] (3,0) -- ++(0, 0.5);

              \draw[dashed] (P1) -- ++(0, 0.5);
              \draw[dashed] (P6) -- ++(0, 0.5);
              \draw[dashed] (P5) -- ++(0, 0.5);
              \draw[dashed] (P7) -- ++(0, 0.5);
              % \draw[<->,thick] (0,0.3) -- (3,0.3) node[midway, above] { ?};
              % \draw[<->,thick] (3,0.3) -- (6,0.3) node[midway, above] { ?};
              % \draw[<->,thick] (6,0.3) -- (9,0.3) node[midway, above] { ?};
              \draw[dashed] (0,-1) -- ++(0, 0.5);
              \draw[dashed] (3,-1) -- ++(0, 0.3);
              \draw[dashed] (6,-1) -- ++(0, 0.3);
              \draw[dashed] (9,-1) -- ++(0, 0.3);
              \draw[dashed] (12,-1) -- ++(0, 0.5);
              \draw[dashed] (11,-1) -- ++(0, 0.3);
              \draw[decorate, decoration={brace, amplitude=10pt}] (11,-1) -- (0,-1)
              node[midway, below=8pt] {30};
              \draw[<->,thick] (0,-0.7) -- (12,-0.7) node[midway, above] { $4x$};
              \draw[decorate, decoration={brace, amplitude=10pt}] (12,-1) -- (11,-1)
              node[midway, below=8pt] {2};

          \end{tikzpicture}

          \vspace{0.5cm}

          $30$ + 2 = 32 \newline
          \indent 1 + 3 = 4 \newline
          \indent $32\div4$  = 8(个) \newline
          \indent $3\times8$ - 2 = 22(个) \newline
          答:小红有22个苹果。\newline
          \textbf{列方程:} \newline
          \indent 设小明有x个苹果,则小红有3x-2个苹果。\newline
          \indent 根据题意可得:x + (3x-2) = 30 \newline
          \indent 4x = 32 \newline
          \indent x = 8 \newline
          \indent 小红有3x-2 = 22个苹果。\newline

    \item Problem 3
          \newline
          \noindent 小明和小红共有37张卡片,其中小明比小红的6倍多2张,那么小明有几张 \newline
          \textbf{解:}
          \begin{tikzpicture}[scale=0.8, >=stealth]
              % 定义两个点
              \coordinate (P1) at (0,0);
              \coordinate (P2) at (1,0);
              \coordinate (P3) at (0,1);
              \coordinate (P4) at (1,1);
              \coordinate (P5) at (2,0);
              \coordinate (P6) at (3,0);
              \coordinate (P7) at (4,0);
              \coordinate (P8) at (5,0);
              \coordinate (P9) at (6,0);
              \coordinate (P10) at (6.5,0);

              % 1. 在起点 (0,0) 处放置文字 (使用 'left' 定位)
              % 首先定义起点 (0,0) 为一个节点 N1,并放置文字
              \node (label1) at (P1) [left] {小明};
              \node (label2) at (P3) [left] {小红};
              \node (label2) at (0,-1) [left] {小明+小红};
              % 然后从这个点开始画线段
            %   \draw[ thick] (P1) -- (P10);
              \draw[ thick] (P3) -- (P4);
              \draw[thick,red] (P1) -- (P10);
              \draw[thick] (0,-1) -- (1,-1);
              \draw[thick,red] (1,-1) -- (7.5,-1);
              % \draw[decorate, decoration={brace, amplitude=7pt, mirror}] (P1) -- (P2)
              % node[midway, below=8pt] {35}; % mirror 选项使大括号朝向P1和P2定义的路径的内部
              \draw[decorate, decoration={brace, amplitude=5pt}] (P3) -- (P4)
              node[midway, above=6pt] {$x$};
              \draw[decorate, decoration={brace, amplitude=5pt}] (8,2) -- (8,0)
              node[midway, right=6pt] {37};



              \draw[dashed] (P1) -- ++(0, 0.3);
              \draw[dashed] (P2) -- ++(0, 0.3);
              \draw[dashed] (P6) -- ++(0, 0.3);
              \draw[dashed] (P7) -- ++(0, 0.3);
              \draw[dashed] (P8) -- ++(0, 0.3);
              \draw[dashed] (P9) -- ++(0, 0.3);
              \draw[dashed] (P10) -- ++(0, 0.3);
              \draw[dashed] (P5) -- ++(0, 0.3);
              \draw[decorate, decoration={brace, amplitude=12pt}] (P1) -- (P9)
              node[midway, above=9pt] {$6x$};
              \draw[decorate, decoration={brace, amplitude=10pt}] (P9) -- (P10)
              node[midway, above=8pt] {2};
              % \draw[<->,thick] (0,0.3) -- (3,0.3) node[midway, above] { ?};
              % \draw[<->,thick] (3,0.3) -- (6,0.3) node[midway, above] { ?};
              % \draw[<->,thick] (6,0.3) -- (9,0.3) node[midway, above] { ?};
              \draw[dashed] (0,-1) -- ++(0, 0.5);
              \draw[dashed] (1,-1) -- ++(0, 0.5);
              \draw[dashed] (2,-1) -- ++(0, 0.5);
              \draw[dashed] (3,-1) -- ++(0, 0.5);
              \draw[dashed] (4,-1) -- ++(0, 0.5);
              \draw[dashed] (5,-1) -- ++(0, 0.5);
              \draw[dashed] (6,-1) -- ++(0, 0.5);
              \draw[dashed] (7,-1) -- ++(0, 0.5);
              \draw[dashed] (7.5,-1) -- ++(0, 0.5);
              \draw[decorate, decoration={brace, amplitude=10pt}] (7.5,-1) -- (0,-1)
              node[midway, below=8pt] {37};


          \end{tikzpicture}
          \vspace{0.5cm}

          $37$ - 2 = 35(个) \newline
          \indent $6$ + 1  = 7 \newline
          \indent $35\div7$  = 5(个) \newline
          \indent $5\times6$ + 2  = 32(个) \newline
          答:小明有32张。\newline
          \newline
          \textbf{列方程:} \newline
          \indent 设小红有x张,则小红有6x+2个苹果。\newline
          \indent 根据题意可得:x + 6x+2 = 37 \newline
          \indent 7x = 35 \newline
          \indent x = 5 \newline
          \indent 小明有6x+2 = 32。\newline
    \item Problem 4
          \newline
          \noindent 小明和小红共有37张卡片,其中小明比小红多3张,那么小明有几张 \newline
          \textbf{解:}
          \begin{tikzpicture}[scale=0.8, >=stealth]
              % 定义两个点
              \coordinate (P1) at (0,0);
              \coordinate (P2) at (1,0);
              \coordinate (P3) at (0,1);
              \coordinate (P4) at (4,1);




              \coordinate (P9) at (4,0);
              \coordinate (P10) at (5,0);

              % 1. 在起点 (0,0) 处放置文字 (使用 'left' 定位)
              % 首先定义起点 (0,0) 为一个节点 N1,并放置文字
              \node (label1) at (P1) [left] {小明};
              \node (label2) at (P3) [left] {小红};
              \node (label2) at (0,-1) [left] {小明+小红};
              % 然后从这个点开始画线段
              \draw[ thick] (P1) -- (P10);
              \draw[ thick] (P3) -- (P4);

              \draw[thick,red] (0,-1) -- (9,-1);
              % \draw[decorate, decoration={brace, amplitude=7pt, mirror}] (P1) -- (P2)
              % node[midway, below=8pt] {35}; % mirror 选项使大括号朝向P1和P2定义的路径的内部
              \draw[decorate, decoration={brace, amplitude=5pt}] (P3) -- (P4)
              node[midway, above=6pt] {$x$};
              \draw[decorate, decoration={brace, amplitude=5pt}] (8,2) -- (8,0)
              node[midway, right=6pt] {37};



              \draw[dashed] (P1) -- ++(0, 0.3);




              \draw[dashed] (P9) -- ++(0, 0.3);
              \draw[dashed] (P10) -- ++(0, 0.3);

              \draw[decorate, decoration={brace, amplitude=12pt}] (P1) -- (P9)
              node[midway, above=9pt] {$x$};
              \draw[decorate, decoration={brace, amplitude=10pt}] (P9) -- (P10)
              node[midway, above=8pt] {3};
              % \draw[<->,thick] (0,0.3) -- (3,0.3) node[midway, above] { ?};
              % \draw[<->,thick] (3,0.3) -- (6,0.3) node[midway, above] { ?};
              % \draw[<->,thick] (6,0.3) -- (9,0.3) node[midway, above] { ?};
              \draw[dashed] (0,-1) -- ++(0, 0.5);


              \draw[dashed] (4,-1) -- ++(0, 0.5);

              \draw[dashed] (8,-1) -- ++(0, 0.5);

              \draw[dashed] (9,-1) -- ++(0, 0.5);
              \draw[decorate, decoration={brace, amplitude=10pt}] (9,-1) -- (0,-1)
              node[midway, below=8pt] {37};
              \draw[<->,thick] (0,-0.7) -- (8,-0.7) node[midway, above] { $2x$};
              \draw[<->,thick] (9,-0.7) -- (8,-0.7) node[midway, above] { 3};

          \end{tikzpicture}
          \vspace{0.5cm}


          $37$ - 3 = 34(个) \newline
          \indent $1$ + 1  = 2 \newline
          \indent $34\div2$  = 17(个) \newline
          \indent $17$ + 3  = 20(个) \newline
          答:小明有20个。\newline
          \newline
          \textbf{列方程:} \newline
          \indent 设小红有$x$张,则小红有$x$+3个苹果。\newline
          \indent 根据题意可得:$x$ + ($x$+3) = 37 \newline
          \indent $2x$ = 34 \newline
          \indent $x$ = 17 \newline
          \indent 小明有$x$+3 = 20。\newline
    \item Problem 5
          \newline
          \noindent 小明和小红共有37张卡片,其中小明比小红少3张,那么小明有几张 \newline
          \textbf{解:}
          \begin{tikzpicture}[scale=0.8, >=stealth]
              % 定义两个点
              \coordinate (P1) at (0,0);
              \coordinate (P2) at (1,0);
              \coordinate (P3) at (0,1);
              \coordinate (P4) at (4,1);




              \coordinate (P9) at (4,0);
              \coordinate (P10) at (3,0);

              % 1. 在起点 (0,0) 处放置文字 (使用 'left' 定位)
              % 首先定义起点 (0,0) 为一个节点 N1,并放置文字
              \node (label1) at (P1) [left] {小明};
              \node (label2) at (P3) [left] {小红};
              \node (label2) at (0,-1) [left] {小明+小红};
              % 然后从这个点开始画线段
              \draw[ thick] (P1) -- (P10);
              \draw[dashed, thick] (P10) -- (P9);
              \draw[ thick] (P3) -- (P4);

              \draw[thick,red] (0,-1) -- (7,-1);
              \draw[thick,dashed] (8,-1) -- (7,-1);
              % \draw[decorate, decoration={brace, amplitude=7pt, mirror}] (P1) -- (P2)
              % node[midway, below=8pt] {35}; % mirror 选项使大括号朝向P1和P2定义的路径的内部
              \draw[decorate, decoration={brace, amplitude=5pt}] (P3) -- (P4)
              node[midway, above=6pt] {$x$};
              \draw[decorate, decoration={brace, amplitude=5pt}] (8,2) -- (8,0)
              node[midway, right=6pt] {37};



              \draw[dashed] (P1) -- ++(0, 0.3);




              \draw[dashed] (P9) -- ++(0, 0.3);
              \draw[dashed] (P10) -- ++(0, 0.3);

              \draw[decorate, decoration={brace, amplitude=12pt}] (P1) -- (P10)
              node[midway, above=9pt] {$x$-3};
              %   \draw[decorate, decoration={brace, amplitude=10pt}] (P9) -- (P10)
              %   node[midway, above=8pt] {3};
              % \draw[<->,thick] (0,0.3) -- (3,0.3) node[midway, above] { ?};
              % \draw[<->,thick] (3,0.3) -- (6,0.3) node[midway, above] { ?};
              % \draw[<->,thick] (6,0.3) -- (9,0.3) node[midway, above] { ?};
              \draw[dashed] (0,-1) -- ++(0, 0.5);


              \draw[dashed] (4,-1) -- ++(0, 0.5);

              \draw[dashed] (8,-1) -- ++(0, 0.5);


              \draw[decorate, decoration={brace, amplitude=10pt}] (7,-1) -- (0,-1)
              node[midway, below=8pt] {37};
              \draw[<->,thick] (0,-0.7) -- (8,-0.7) node[midway, above] { $2x$};

              \draw[decorate, decoration={brace, amplitude=10pt}] (8,-1) -- (7,-1)
              node[midway, below=8pt] {3};

          \end{tikzpicture}
          \vspace{0.5cm}


          $37$ + 3 = 40(个) \newline
          \indent $1$ + 1  = 2 \newline
          \indent $40\div2$  = 20(个) \newline
          \indent $20$ - 3  = 17(个) \newline
          答:小明有17个。\newline
          \newline
          \textbf{列方程:} \newline
          \indent 设小红有$x$张,则小红有$x$-3个苹果。\newline
          \indent 根据题意可得:$x$ + ($x$-3) = 37 \newline
          \indent $2x$ = 40 \newline
          \indent $x$ = 20 \newline
          \indent 小明有$x$-3 = 17。\newline

          \textbf{后面两个题目其实本质上与前两个是一样的,只是1倍。}
\end{itemize}

\subsubsection{差倍}

\begin{itemize}
    \item Problem 1
          \newline
          \noindent 小红比小明多36个苹果,小红苹果数量是小明的3倍。小红有多少个苹果? \newline
          \textbf{解:}
          \begin{tikzpicture}[scale=0.8, >=stealth]
              % 定义两个点
              \coordinate (P1) at (0,0);
              \coordinate (P2) at (3,0);
              \coordinate (P3) at (0,1);
              \coordinate (P4) at (3,1);
              \coordinate (P5) at (9,0);
              \coordinate (P6) at (6,0);
              % 1. 在起点 (0,0) 处放置文字 (使用 'left' 定位)
              % 首先定义起点 (0,0) 为一个节点 N1,并放置文字
              \node (label1) at (P1) [left] {小红};
              \node (label2) at (P3) [left] {小明};
              \node (label2) at (0,-1) [left] {小红-小明};
              % 然后从这个点开始画线段
              \draw[ thick] (P1) -- (P2);
              \draw[ thick] (P3) -- (P4);
              \draw[thick,red] (P1) -- (P5);
              \draw[thick,red] (0,-1) -- (6,-1);
              % \draw[decorate, decoration={brace, amplitude=7pt, mirror}] (P1) -- (P2)
              % node[midway, below=8pt] {35}; % mirror 选项使大括号朝向P1和P2定义的路径的内部
              \draw[decorate, decoration={brace, amplitude=5pt}] (P3) -- (P4)
              node[midway, above=6pt] {x};
              \draw[decorate, decoration={brace, amplitude=5pt}] (3,0.5) -- (9,0.5)
              node[midway, right=6pt] {36};

              \draw[dashed] (P1) -- ++(0, 0.5);
              \draw[dashed] (3,0) -- ++(0, 0.5);

              \draw[dashed] (P1) -- ++(0, 0.5);
              \draw[dashed] (P6) -- ++(0, 0.5);
              \draw[dashed] (P5) -- ++(0, 0.5);
              % \draw[<->,thick] (0,0.3) -- (3,0.3) node[midway, above] { ?};
              % \draw[<->,thick] (3,0.3) -- (6,0.3) node[midway, above] { ?};
              % \draw[<->,thick] (6,0.3) -- (9,0.3) node[midway, above] { ?};
              \draw[dashed] (0,-1) -- ++(0, 0.5);
              \draw[dashed] (3,-1) -- ++(0, 0.5);
              \draw[dashed] (6,-1) -- ++(0, 0.5);


              \draw[decorate, decoration={brace, amplitude=10pt}] (6,-1) -- (0,-1)
              node[midway, below=8pt] {36};

          \end{tikzpicture}

          \vspace{0.5cm}

          $3$ - 1 = 2 \newline
          \indent $36\div2$  = 18(个) \newline
          \indent $3\times18$  = 54(个) \newline

          答:小红有54个苹果。\newline
          \textbf{列方程:} \newline
          \indent 设小明有$x$个苹果,则小红有$3x$个苹果。\newline
          \indent 根据题意可得:$3x$ - $x$ = 36 \newline
          \indent $2x$ = 36 \newline
          \indent $x$ = 918 \newline
          \indent 小红有$3x$ = 54个苹果。\newline
    \item Problem 2
          \newline
          \noindent 小红比小明多25个苹果,小红苹果个数比小明的三倍多5个。小红有多少个苹果? \newline
          \textbf{解:}
          \begin{tikzpicture}[scale=0.8, >=stealth]
              % 定义两个点
              \coordinate (P1) at (0,0);
              \coordinate (P2) at (3,0);
              \coordinate (P3) at (0,1);
              \coordinate (P4) at (3,1);
              \coordinate (P5) at (9,0);
              \coordinate (P6) at (6,0);
              \coordinate (P7) at (10,0);
              % 1. 在起点 (0,0) 处放置文字 (使用 'left' 定位)
              % 首先定义起点 (0,0) 为一个节点 N1,并放置文字
              \node (label1) at (P1) [left] {小红};
              \node (label2) at (P3) [left] {小明};
              \node (label2) at (0,-1) [left] {小红-小明};
              % 然后从这个点开始画线段
              \draw[ thick] (P1) -- (P2);
              \draw[ thick] (P3) -- (P4);
              \draw[thick,red] (P1) -- (P7);
              \draw[thick,red] (0,-1) -- (7,-1);
              % \draw[decorate, decoration={brace, amplitude=7pt, mirror}] (P1) -- (P2)
              % node[midway, below=8pt] {35}; % mirror 选项使大括号朝向P1和P2定义的路径的内部
              \draw[decorate, decoration={brace, amplitude=5pt}] (P3) -- (P4)
              node[midway, above=6pt] {$x$};
              \draw[decorate, decoration={brace, amplitude=5pt}] (0,0.5) -- (9,0.5)
              node[midway, above=6pt] {$3x$};
              \draw[decorate, decoration={brace, amplitude=5pt}] (9,0.5) -- (10,0.5)
              node[midway, above=6pt] {5};

              \draw[dashed] (P1) -- ++(0, 0.5);
              \draw[dashed] (3,0) -- ++(0, 0.5);

              \draw[dashed] (P1) -- ++(0, 0.5);
              \draw[dashed] (P6) -- ++(0, 0.5);
              \draw[dashed] (P5) -- ++(0, 0.5);
              \draw[dashed] (P7) -- ++(0, 0.5);
              % \draw[<->,thick] (0,0.3) -- (3,0.3) node[midway, above] { ?};
              % \draw[<->,thick] (3,0.3) -- (6,0.3) node[midway, above] { ?};
              % \draw[<->,thick] (6,0.3) -- (9,0.3) node[midway, above] { ?};
              \draw[dashed] (0,-1) -- ++(0, 0.5);
              \draw[dashed] (3,-1) -- ++(0, 0.5);
              \draw[dashed] (6,-1) -- ++(0, 0.5);
              \draw[dashed] (7,-1) -- ++(0, 0.5);


              \draw[decorate, decoration={brace, amplitude=10pt}] (7,-1) -- (0,-1)
              node[midway, below=8pt] {25};

          \end{tikzpicture}

          \vspace{0.5cm}

          $3$ - 1 = 2 \newline
          $25$ - 5 = 20 \newline
          \indent $20\div2$  = 10(个) \newline
          \indent $3\times10$ + 5  = 35(个) \newline

          答:小红有35个苹果。\newline
          \textbf{列方程:} \newline
          \indent 设小明有$x$个苹果,则小红有$3x+5$ 或者$x+25$个苹果。\newline
          \indent 根据题意可得:$3x+5$ = $x+25$ \newline
          \indent $2x$ = 20 \newline
          \indent $x$ = 10 \newline
          \indent 小红有$x$ +25 = 35个苹果。\newline
    \item Problem 3
          \newline
          \noindent 小红苹果个数是小明的5倍,如果小红给小明10个苹果,那么两人苹果个数就一样多。小红有多少个苹果? \newline
          \textbf{解:}
          \begin{tikzpicture}[scale=0.8, >=stealth]
              % 定义两个点
              \coordinate (P1) at (0,0);
              \coordinate (P2) at (2,0);
              \coordinate (P3) at (0,1);
              \coordinate (P4) at (2,1);
              \coordinate (P5) at (4,0);
              \coordinate (P6) at (6,0);
              \coordinate (P7) at (8,0);
              \coordinate (P8) at (10,0);
              % 1. 在起点 (0,0) 处放置文字 (使用 'left' 定位)
              % 首先定义起点 (0,0) 为一个节点 N1,并放置文字
              \node (label1) at (P1) [left] {小红};
              \node (label2) at (P3) [left] {小明};
              \node (label2) at (0,-1) [left] {小红-小明};
              % 然后从这个点开始画线段
              \draw[ thick] (P1) -- (P8);
              \draw[ thick] (P3) -- (P4);

              \draw[thick,red] (0,-1) -- (4,-1);
              % \draw[decorate, decoration={brace, amplitude=7pt, mirror}] (P1) -- (P2)
              % node[midway, below=8pt] {35}; % mirror 选项使大括号朝向P1和P2定义的路径的内部
              \draw[decorate, decoration={brace, amplitude=5pt}] (P3) -- (P4)
              node[midway, above=6pt] {$x$};
              \draw[decorate, decoration={brace, amplitude=5pt}] (0,0.5) -- (10,0.5)
              node[midway, above=6pt] {$5x$};





              \draw[dashed] (P1) -- ++(0, 0.5);
              \draw[dashed] (P2) -- ++(0, 0.5);
              \draw[dashed] (P6) -- ++(0, 0.5);
              \draw[dashed] (P5) -- ++(0, 0.5);
              \draw[dashed] (P7) -- ++(0, 0.5);
              \draw[dashed] (P8) -- ++(0, 0.5);
              % \draw[<->,thick] (0,0.3) -- (3,0.3) node[midway, above] { ?};
              % \draw[<->,thick] (3,0.3) -- (6,0.3) node[midway, above] { ?};
              % \draw[<->,thick] (6,0.3) -- (9,0.3) node[midway, above] { ?};
              \draw[dashed] (0,-1) -- ++(0, 0.5);
              \draw[dashed] (2,-1) -- ++(0, 0.5);
              \draw[dashed] (4,-1) -- ++(0, 0.5);



              \draw[decorate, decoration={brace, amplitude=10pt}] (4,-1) -- (0,-1)
              node[midway, below=8pt] {10};

          \end{tikzpicture}

          \vspace{0.5cm}

          $5$ - 1 = 4 \newline
          \indent $4\div2$  =  \newline
          \indent $10\div2$  = 5(个) \newline
          \indent $3\times5$   = 15(个) \newline

          答:小红有15个苹果。\newline
          \textbf{列方程:} \newline
          \indent 设小明有$x$个苹果,则小红有$5x$ 或者$x+25$个苹果。\newline
          \indent 根据题意可得:$5x-10$ = $x+10$ \newline
          \indent $4x$ = 20 \newline
          \indent $x$ = 5 \newline
          \indent 小红有$5x$  = 20个苹果。\newline


\end{itemize}




\noindent   注意\footnote{本讲义同样适合初中学生使用,比如列方程等。}:以上所有问题均可以通过列方程的方法进行解答,建议学生尝试用列方程的方法进行解答。:



\end{problem}
%---------------------------------------------------------------------------------------------
% \begin{problem}

% \section{Segundo Problema}

% \noindent

% \end{problem}
%---------------------------------------------------------------------------------------------



%--------------------------------------BIBLIOGRAFIA-------------------------------------------

\newpage

\nocite{*} % Agrega las referencias aunque no las hayas citado directamente

\bibliographystyle{unsrt}    % ESTILO DE BIBLIOGRAFÍA (Recomendados: abbrv, ieeetr, apalike, unsrt)
\bibliography{refs}     % REFERENCIAS EN ARCHIVO SEPARADO


\end{document}
