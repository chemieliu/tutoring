\documentclass[12pt]{article}
\usepackage[UTF8]{ctex}  % 中文支持

\usepackage[utf8]{inputenc}
\usepackage{amsmath}
\usepackage{amssymb}
\usepackage{geometry}
\usepackage{graphicx}
\usepackage{amsmath, amssymb}
\usepackage{geometry}
\usepackage{tikz}

\geometry{a4paper, margin=1in}

% \title{Homework 1206}
% \author{Liu}
% \date{\today}

\begin{document}

% \maketitle


\section*{Problem 1}
已知一次函数 $y = ax - 3 - a$(其中 $a \ne 0$)。

\begin{enumerate}
    \item[(1)] 当 $y = -3$ 时,$x =$\underline{\hspace{2cm}};
    \item[(2)] 当 $-4 \leq y \leq -1$ 时,自变量 $x$ 的\textbf{负整数值}恰好有 $2$ 个,则 $a$ 的取值范围为\underline{\hspace{4cm}}。
\end{enumerate}
\section*{Problem 2}
甲、乙两人骑自行车匀速同向行驶,乙在甲前面 $300\,\text{m}$ 处,同时出发去距离甲 $1200\,\text{m}$ 的目的地,甲的速度比乙快。设甲、乙之间的距离为 $y$(单位:米),乙行驶的时间为 $x$(单位:秒),$y$ 与 $x$ 之间的关系如图所示,则点 $C$ 的坐标为(\underline{\hspace{2cm}})。

\begin{itemize}
    \item[A.] $(200,\ 160)$
    \item[B.] $(200,\ 180)$
    \item[C.] $(240,\ 160)$
    \item[D.] $(240,\ 180)$
\end{itemize}
\begin{center}
    \begin{tikzpicture}[scale=0.02]
        % 坐标轴
        \draw[->] (0,0) -- (330,0) node[right] {$x\ (\text{s})$};
        \draw[->] (0,0) -- (0,350) node[above] {$y\ (\text{m})$};

        % 标记点
        \foreach \x in {  150,  300}
        \draw[shift={(\x,0)}] (0,0) -- (0,3) node[below=3pt] {\x};
        \foreach \y in {  300}
        \draw[shift={(0,\y)}] (0,0) -- (3,0) node[left=3pt] {\y};

        % 折线
        \draw[thick] (0,300) node[right] {A} -- (150,0) node[above right] {B} -- (225,150) node[above] {C} -- (300,0) node[above right] {D};

        % 原点
        \filldraw (0,0) circle (2pt) node[below left] {O};
        \filldraw (0,300) circle (2pt);
        \filldraw (150,0) circle (2pt);
        \filldraw (225,150) circle (2pt);
        \filldraw (300,0) circle (2pt);
    \end{tikzpicture}
\end{center}

\section*{Problem 2}

\textbf{10.}
如图,点 $B,\ C,\ D$ 在同一直线上,$\triangle DCE$ 沿 $CE$ 折叠,点 $D$ 恰好落在直角三角形 $\triangle ABC$ 的直角顶点 $A$ 处。若 $\angle D = 45^\circ$,$DE = 2$,则 $BD - AB$ 的值为(\underline{\hspace{2cm}})。

\begin{itemize}
    \item[A.] $2$
    \item[B.] $2\sqrt{2}$
    \item[C.] $4$
    \item[D.] $4\sqrt{2}$
\end{itemize}
\begin{figure}[h]
    \centering
    \begin{tikzpicture}[scale=1]
        
        % 1. 定义顶点坐标
        % 假设 C 为原点,B, C, D 共线
        \coordinate (B) at (-5, 0);
        \coordinate (C) at (0, 0);
        \coordinate (D) at (3, 0);
        \coordinate (A) at (-1.8, 2.4); % 调整 A 的位置使其看起来像原图
        \coordinate (E) at (1.1, 1.6);
        
        % 2. 绘制实线段
        \draw (A) -- (B);  % AB
        \draw (A) -- (C);  % AC
        \draw (A) -- (E);  % AE
        \draw (C) -- (E);  % CE
        \draw (B) -- (C);  % BC (基线的一部分)
        
        % 3. 绘制虚线段 (dashed)
        \draw[dashed] (C) -- (D);  % CD (基线的剩余部分)
        \draw[dashed] (E) -- (D);  % ED
        
        % 4. 标注顶点
        \node[above] at (A) {$A$};
        \node[below left] at (B) {$B$};
        \node[below] at (C) {$C$};
        \node[below right] at (D) {$D$};
        \node[above right] at (E) {$E$};
        
    \end{tikzpicture}
    % \caption{几何图形示意图} % 如果需要添加图注,请取消注释
\end{figure}


\end{document}