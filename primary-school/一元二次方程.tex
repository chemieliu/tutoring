\documentclass[12pt]{article}
\usepackage[UTF8]{ctex}  % 中文支持

\usepackage[utf8]{inputenc}
\usepackage{amsmath}
\usepackage{amssymb}
\usepackage{geometry}
\usepackage{graphicx}
\usepackage{amsmath, amssymb}
\usepackage{geometry}
\usepackage{tikz}

% Define the \stars command
\newcommand{\stars}[1]{\textbf{#1}★}

\geometry{a4paper, margin=1in}

% \title{Homework 1206}
% \author{Liu}
% \date{\today}

\begin{document}


\section*{一元二次方程}


\section{什么是一元二次方程}

一元二次方程是只含有一个未知数,并且未知数的最高次数为二的方程。
其一般形式为:


\[
  ax^2 + bx + c = 0 \qquad (a \neq 0)
\]


其中,\(a, b, c\) 为已知常数,\(x\) 为未知数。

\section{判别式}

为了判断方程是否有实数解,我们使用判别式:


\[
  \Delta = b^2 - 4ac
\]



根据判别式的值,可以判断方程解的情况:
\begin{itemize}
  \item 若 \(\Delta > 0\),方程有两个不相等的实数根;
  \item 若 \(\Delta = 0\),方程有两个相等的实数根;
  \item 若 \(\Delta < 0\),方程无实数根(有两个共轭复数根)。
\end{itemize}

\section{求根公式}

当 \(a \neq 0\) 时,一元二次方程的解可由求根公式给出:


\[
  x = \frac{-b \pm \sqrt{b^2 - 4ac}}{2a}
\]



\section{示例}

求解方程:


\[
  2x^2 - 3x - 2 = 0
\]



计算判别式:


\[
  \Delta = (-3)^2 - 4 \cdot 2 \cdot (-2) = 9 + 16 = 25
\]



代入求根公式:


\[
  x = \frac{3 \pm \sqrt{25}}{4}
\]




\[
  x_1 = 2,\qquad x_2 = -\frac{1}{2}
\]

\section{韦达定理}

对于一元二次方程


\[
  ax^2 + bx + c = 0 \qquad (a \neq 0),
\]


设方程的两个实根为 \(x_1, x_2\),则有:



\[
  x_1 + x_2 = -\frac{b}{a},
\]




\[
  x_1 x_2 = \frac{c}{a}.
\]



特别地,当方程为


\[
  x^2 + bx + c = 0,
\]


则有:


\[
  x_1 + x_2 = -b,\qquad x_1 x_2 = c.
\]



\section*{例题讲解}

\textbf{例1} 把下列方程化为一元二次方程的一般形式,并指出二次项系数、一次项系数和常数项。

\subsection*{(1) 原式:\quad \( x(x - 2) = 4x^2 - 3x \)}

\begin{align*}
  \text{左边展开:}  & \quad x^2 - 2x                 \\
  \text{整理方程:}  & \quad x^2 - 2x = 4x^2 - 3x     \\
  \text{移项:}    & \quad x^2 - 2x - 4x^2 + 3x = 0 \\
  \text{合并同类项:} & \quad -3x^2 + x = 0
\end{align*}

\textbf{一般形式:} \quad \( -3x^2 + x = 0 \)

\textbf{二次项系数:} \(-3\)
\textbf{一次项系数:} \(1\)
\textbf{常数项:} \(0\)

\vspace{1em}

\subsection*{(2) 原式:\quad \( \frac{x^2}{3} + \frac{x + 1}{2} = \frac{-x - 1}{2} \)}

\begin{align*}
  \text{左边整理:}   & \quad \frac{x^2}{3} + \frac{x}{2} + \frac{1}{2}                                 \\
  \text{右边:}     & \quad -\frac{x}{2} - \frac{1}{2}                                                \\
  \text{移项:}     & \quad \frac{x^2}{3} + \frac{x}{2} + \frac{1}{2} + \frac{x}{2} + \frac{1}{2} = 0 \\
  \text{合并:}     & \quad \frac{x^2}{3} + x + 1 = 0                                                 \\
  \text{通分:}     & \quad \frac{x^2 + 3x + 3}{3} = 0                                                \\
  \text{两边乘以 3:} & \quad x^2 + 3x + 3 = 0
\end{align*}

\textbf{一般形式:} \quad \( x^2 + 3x + 3 = 0 \)

\textbf{二次项系数:} \(1\)
\textbf{一次项系数:} \(3\)
\textbf{常数项:} \(3\) \newline


\textbf{例2} 若 \(x = 1\) 是关于 \(x\) 的一元二次方程


\[
  x^2 + mx - 6 = 0
\]


的一个根,求 \(m\) 的值。
\newline
\textbf{解:}
将 \(x = 1\) 代入方程:
\[
  1^2 + m \cdot 1 - 6 = 0
\]
\[
  1 + m - 6 = 0
\]
\[
  m - 5 = 0
\]
因此,\(m = 5\)。\newline
\textbf{例3} 若 \(m\) 是关于 \(x\) 的一元二次方程


\[
  x^2 -2x - 1 = 0
\]


的一个根,求 \[
  m^2+\frac{1}{m^2}  \]的值。
\newline
\textbf{解:}
将 \(x = m\) 代入方程:
\[
  m - 2 - \frac{1}{m} = 0
\]

\[
  m - \frac{1}{m} = 2
\]

\[
  \left(m - \frac{1}{m}\right)^2 = 4
\]

\[
  m^2 - 2 + \frac{1}{m^2} = 4
\]

\[
  m^2 + \frac{1}{m^2} = 6
\] \newline
\textbf{例4} 已知关于 \(x\) 的一元二次方程


\[
  x^2 +2mx +m^2 - 1 = 0
\]

判断方程根的情况。
\newline
\textbf{例5} 若 $\alpha, \beta$ 是关于 一元二次方程
\[
  3x^2 +2x -9 = 0
\]
的两根,则


\[
  \frac{\beta}{\alpha} + \frac{\alpha}{\beta} =
\]
\end{document}